\chapter{Discussion}
\label{chap:discuss}

\section{Selection of haemocyte medium for flow cytometric analyses}
Even though the aggregation model was slightly over-dispersed, the estimates provide some insight into the three buffers relative abilities to prevent hemocyte aggregation within the first hour post-withdrawal. The combination of Ca$^{2+}$-free and \acrshort{edta}-containing buffers were effective inhibitors of hemocyte aggregation compared to simply diluting samples on ice. When the latter method was used, visible aggregates were usually formed within the syringes immediately after hemolymph aspiration - even though the \acrshort{mpss} was pre-chilled on ice. This observation shows that a two-fold dilution in MPSS is less than sufficient, such that a further dilution may be required for satisfactory effect. A many-fold dilution would be inconvenient in the preparation of haemolymph smears with a certain desired density, and too time-consuming for the acquisition of 10.000 events on a flow cytometer. This approach was therefore ruled out of question.

By comparing the inhibitory effects of \acrshort{mas} and \acrshort{acb} on haemocyte aggregation, our data suggests the slightly higher concentration of \acrshort{edta} in \acrshort{acb} compensates for the lack of citrate. Since high concentrations of \acrshort{edta} have been reported to impair haemocyte viability (\cite{Grandiosa2018, Burkhard2009}), a direct comparison of \acrshort{mas} and \acrshort{acb} with regards to acute effects on viability was required to identify the most suitable buffer of the two.

Our results suggest that \acrshort{edta} is cytotoxic to haemocytes at the concentrations used in both \acrshort{acb} (13.4 mM) and \acrshort{mas} (11.5 mM), since these buffers caused a significant dose-dependent increase in the percentage of necrotic haemocytes across the three timepoints (Table \ref{tb:Paired_ttests}). However, since there was no significant differences between the \acrshort{edta}-containing buffers and the negative control group after 15 minutes of incubation, this cytotoxic effect had no detectable manifestation within the time-frame of the planned flow cytometric assay. As long as haemolymph samples are stained and processed within 30 minutes of sampling, a flow cytometric dye exclusion/inclusion assay with TO-PRO$^{TM}$-3 Iodide and \acrshort{calceinam} will not have time to detect \emph{in vitro} necrosis caused by the buffers themselves.

The apoptosis assay with non-adjusted MAS (pH = 6.1) showed that 15 minutes was more than enough for haemocytes to enter programmed cell death. The percentage of cells already in late apoptosis suggests that an abrupt decrease in pH is an efficacious inducer of apoptosis, and that the haemocytes of \emph{M. edulis} are very sensitive to the environmental pH. This has been demonstrated by several recent studies investigating the effects of ocean acidification on the immune system of bivalves (see e.g., \cite{Wang2016, Dang2023}). Maintaining the pH of MAS at 6.1 is therefore not an option for flow cytometric analyses of apotosis or necrosis.

Our analyses did not detect any differences between ACB and MAS (pH = 7.0) with regard to their anticoagulant effects or cytotoxicity within the time-frame of the planned assays. The data does however demonstrate the importance of a carefully regulated haemocyte medium pH. As the buffer capacity of pH-adjusted MAS is negligible, ACB appears as the most suitable haemocyte medium of the two. The Anticoagulant Buffer (ACB) of Pipe et al. (1997) was therefore used for flow cytometric analyses in the Hybrid MN Cytome Assay.

\section{Development of a flow cytometric differential count}
The haemolymph of \emph{M. edulis} were found to contain three morphologically distinct cell types according to traditional cytological criteria. These comprised (1) small agranular basophilic cells (blast-like basophils), (2) larger basophilic cells with small inconspicuous granules (basophilic granulocytes) and (3) large eosinophilic cells with cytoplasm densely populated by larger eosinophilic granules (eosinophilic granulocytes). There has been scientific dispute regarding the classification of basophilic and eosinophilic granulocytes as two distinct cell-types (\cite{Cheng1980}), however; this distinction is made for purely descriptive purposes herein.

Similar to the cytological characterization, a maximum of three distinct subpopulations could be separated according to relative size (FSC) and internal complexity (SSC) in suspensions of living haemocytes. These comprised one subpopulation of small cells with low internal complexity (cluster 1), one subpopulation of larger cells with intermediate internal complexity (\emph{cluster 2}) and one subpopulation of large cells with high internal complexity (cluster 3). The apparent correlation between these subpopulations and the three cell types were striking with regard to their relative sizes and granularity.

The small blast-like basophils (5.63 $\pm{0.72}$ \micro m) were considerably smaller than the basophilic and eosinophilic granulocytes, and exhibited no apparent granulation. One would therefore expect them to be unambiguously separated from the larger granulocytes according to both \acrshort{fsc} and \acrshort{ssc}. The size distributions in Figure \ref{fig:Diameters} does however indicate an overlap in size between the largest blast-like basophils and the smallest basophillic granulocytes in some mussels (see Figure \ref{fig:Diameters}). But, since they are uncomplex cells, they should be separated according to \acrshort{ssc} regardless. The events populating cluster 1 in Figure \ref{fig:fsc_vs_ssc} were therefore expected to represent small blast-like basophilic haemocytes. 

Both eosinophilic and basophilic granulocytes were granulated in the formal definition of the word. But this discussion requires a more nuanced interpretation of what is meant by granulocyte herein. The cytoplasm of eosinophilic granulocytes were packed with pink to dark purple granules to the extent that their cytoplasm appeared pink in a non-spread state. This stands in sharp contrast to the granulation of basophilic granulocytes, which were more sparse, variable and much less conspicuous. Consequently, there is little doubt that cluster 3 is expected to correspond to eosinophilic granulocytes, while the semi-granular events of \emph{\emph{cluster 2}} aligns with the size and complexity of basophilic granulocytes.

As seen from the size distributions in Figure \ref{fig:Diameters}, the basophilic and eosinophilic granulocytes were not readily distinguishable according to cell size. That result was further substantiated by the fact that \emph{cluster 2} and 3 have substantially overlapping \acrshort{fsc}-values. The haemolymph samples depicted in Figure \ref{fig:fsc_vs_ssc} does however show that the \acrshort{fsc} of cluster 3 is slightly right-shifted relative to \emph{cluster 2}, which is expected since the eosinophilic granulocytes were found to be 0.92 \micro M larger than basophilic granulocytes on average.

Even thought there was an apparent correlation between cluster 1-3 and the size and internal complexity of the cytologically defined cell types; these interpretations required visual verification before a potential gating strategy could be implemented from these results. The ispycnic separation of eosinophilic granulocytes from the two basophilic cell types allowed for these cell types to be characterized by FSC vs. SSC separately.

The eosinophilic granulocytes (> 96\%) formed a dense cluster of events with high SSC relative to the two basophilic cell types, and were thus positively unidentified as the cells populating cluster 3 in figure \ref{fig:fsc_vs_ssc}. This observation was further supported by the fact that eosin$^{bright}$ events populated the same cluster when samples of formaldehyde-fixed cells stained with 0.5\% eosin were back-gated to bivariate plots of FSC vs. SSC. Since there was a strong correlation between eosin$^{bright}$ events (\%) and the percentage of eosinophilic granulocytes (R$^{2}$=0.99), this finding also provides solid evidence for this hypothesis. The same conclusion was reached by LeFoll et al., (2010), from which the latter experiment originated.

Basophilic granulocytes and blast-like basophils were not separated according to density by the discontinuous Percoll gradient. However, the basophilic cells that populated the middle fraction of the gradient were separated into two subpopulations according to relative size and granularity by the subsequent flow cytometric analysis (Figure \ref{fig:Percoll-dotplots}C). These subpopulations occupied the regions corresponding to \emph{cluster 1} and \emph{cluster 2} in suspensions of living haemocytes, and eosin$^{dim}$ events in the measurements of eosin fluorescence. Inferred from their relative size and complexity, \emph{cluster 2} corresponds to larger basophilic granulocytes, while \emph{cluster 1} corresponds to small blast-like basophils. [Must be verified by linearity of response by simple linear regression].

The substitution of FSC for calcein fluorescence on the x-axis greatly simplified the process of gating on cluster 1-3 


\section{Scoring of necrotic haemocytes by flow cytometry}






\section{Scoring of apoptotic haemocytes by flow cytometry}
