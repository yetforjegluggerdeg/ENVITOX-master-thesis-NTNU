\chapter{Introduction}

Nanotechnology is key enabling technology of the 21st century with great potential for addressing current societal challenges (EU science hub, 2021, para. 1-2). The technology has already found applications in the major industrial sectors of material manufacturing and electronics and is progressively being employed in the fields of life sciences and health care (\cite{Talebian2021}). The unique material properties that are enhanced or enabled at nanoscale has also led to their introduction into a fast-growing number of household and consumer products, introducing the technology into our homes. The European Commission has defined nanomaterials as “\emph{a natural, incidental or manufactured material containing particles, in an unbound state or as an aggregate or as an agglomerate and where, for 50\% or more of the particles in the number, size distribution, one or more external dimensions is in the size range 1–100 nm}” (European Comission 2012). In consumer products, engineered nanoparticles (ENPs) are added to materials to convey certain physiochemical properties, or they are applied to material surfaces of products to provide desired surface properties such as scratch resistance, water repellency, reflectivity and photo activity (\cite{Bodarenko2013, Weir2012}).

\acrshort{ENPs} are classified according to both chemistry and geometry (\cite{Warheit2018}). In consumer products, metal, and metal oxide (ceramic) isometric particles have found good uses as antimicrobial and/or UV-scattering agents (\cite{Bodarenko2013}). The most common \acrshort{ENPs} in consumer products are metallic silver (\acrshort{Ag}) NPs, with a yearly global production volume of 55 tons (\cite{Piccinno2012}). With 10.000 and 550 metric tons yearly, the metallic oxides titanium(IV)oxide (\acrshort{tio2}) zinc(II)oxide (\ce{ZnO}), respectively, have higher production volumes, but have in turn several other areas of applications (\cite{Piccinno2012, Bodarenko2013}). \acrshort{Ag} NPs is the most widely commercialized antimicrobial NP agent and are especially used in personal care products, sport clothing and washing machines (\cite{Bodarenko2013, Farkas2011}). \ce{TiO2} and \ce{ZnO} NPs are often added to sunscreens and cosmetics for their UV-scattering properties, while {\ce{TiO2}}'s photocatalytic properties at the nanoscale make them effective antimicrobials too (\cite{Bodarenko2013, Weir2012}).

The application of \acrshort{ENPs} in personal care products and fabrics leads to household discharges of NPs into municipal wastewater and sewage streams during the product’s lifecycle. Monitoring influent patterns of twenty elements in the two wastewater treatment plants (\acrshort{wwtp}) of Trondheim city’s catchment (Ladehammeren Renseanlegg, LAD; Høvringen Avløpsrenseeanlegg, HØV), scientists found a cyclic diurnal influent pattern for some of the investigated elements, including Zn, with peaks in the morning and/or the evening (\cite{Farkas2020}). A previous study focusing on the occurrence of nanoparticulate \acrshort{Ag} and \ce{TiO2} in the same \acrshort{wwtp} revealed the same diurnal influent pattern for \ce{TiO2} \acrshort{ENPs}, indicating household contributions to these element discharges (\cite{Polesel2018}). \acrshort{Ag} exhibited more irregular influent profiles, suggesting larger short-term discharges from one or a few point sources (e.g., industry and/or other commercial activity) (\cite{Polesel2018}). 

In full scale \acrshort{wwtp} employing secondary and tertiary treatment steps, removal efficiencies of inorganic elements are predominantly high (> 90\%) (\cite{Cantinho2016}). Most \acrshort{wwtp} employed in smaller communities and cities in Norway, however, only employ preliminary and primary treatment steps (\cite{Berge2018}). This also true for the \acrshort{wwtp} in Trondheim, Norway (\cite{Farkas2020}). The removal efficiencies of \acrshort{Ag} and Ti from the influent wastewater in these catchments are 78±4\% and 81\% at LAR, and 69$\pm$16\% and 84 $\pm$ 4\% at LAD, respectively (\cite{Polesel2018}). The removal efficiency of Zn is even lower, laying somewhere between 50-70\% at both \acrshort{wwtp} (\cite{Farkas2020}). Consequently, substantial amounts dissolved and nanoparticulate \acrshort{Ag}, Ti and Zn enter directly into Trondheimsfjorden after preliminary and primary treatment steps. 

Owing to their size, \acrshort{ENPs} have high surface to volume ratios, and thus exceptionally high reactivity (\cite{Warheit2018}). Their nanoscale metrics also increase their bioavailability compared to microparticles, making their anthropogenic releases and impacts on susceptible marine organisms an important area of study. This is especially true since the already fast-growing field of nanotechnology can expect exponential growth in near future (\cite{Talebian2021}). 

The common blue mussel (\emph{Mytilus edulis}) – a marine benthic invertebrate – resides in the immediate area of wastewater effluent releases. Living in the sediment/water interphase, these suspension feeders filter high quantities of water for suspended particulate matter (\cite{Beyer2017b}). This feeding strategy makes them highly effective in micro- and nano-scaled particle uptake, and consequently especially susceptible to engineered NP exposure (\cite{Canesi2012}). Serving as a marine pollution monitoring species since the 1970s (\cite{Goldberg1975}), this species is a highly suitable sentinel specie for assessing engineered NP toxicity.

On their way through sewage streams and the wastewater treatment process, the coatings and surfaces of \acrshort{ENPs} are impacted – altering their physiochemical properties and behaviour in environmental media (\cite{Kaegi2013}). This “aging” process, in addition to the medium composition, can change their environmental fates, bioavailability and consequently their adverse effects in biota (\cite{Metreveli2016, Georgantzopoulou2020}). Since most laboratory studies within the field of nanotoxicology are performed with pristine \acrshort{ENPs}, there is currently an urge to perform more environmentally realistic exposure experiments to investigate the effects of aged nanoparticles (\cite{Metreveli2016}).



It is our hypothesis that the hemolymph withdrawn from the posterior adductor muscle is withdrawn from blood in the final branches of the left and right posterior \acrshort{gi} arteries (Eggermont, 2020)



Include a separate subsection for the method development part?

Scoring a defined subpopulation of hemocytes for nuclear anomalies by light microscopy is a time-consuming and labor-intensive process... Bridge to semi-automation: inter-operator variability, subjectivity etc.




\footnote{see, e.g., 
\url{https://github.com/COPCSE-NTNU/bachelor-thesis-NTNU} and \url{https://github.com/COPCSE-NTNU/master-theses-NTNU}}
