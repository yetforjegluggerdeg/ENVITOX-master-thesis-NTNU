\chapter{Theory}
\section{The Model Organism/Study Animal}
Euryhaline, osmoconformer, closes valves in periods of exposure to brackish/fresh water (low tide), keeping the saline pallial/mantle fluid as the immediate surrounding environment. Becomes isosmotic with the pallial fluid (Gilles, 1972). In long exposures (> 75 hours) or by puncturing/keeping the valves prised they are forced to pump water, and the hemolymph rapidly conforms to the exterior osmolarity. Short said: is an osmoconformer that behaviorally protects itself from short-term exposures to hypo-osmotic conditions rather than physiologically (Davenport, 1979). Relevant for the osmolarity of buffers/solutions used.

\begin{figure}[H]
    \centering
    \includegraphics[width=\textwidth]{figures/Anatomy/M_edulis_anatomical_axis_lateral.jpg}
    \caption{The figure caption depends on if it ends up here, or in the material and method. Write when decided. The illustration was adapted from an artistic work by Abby Towne, A. Towne Design with permission.}
    \label{fig:anatomical_axis}
\end{figure}
