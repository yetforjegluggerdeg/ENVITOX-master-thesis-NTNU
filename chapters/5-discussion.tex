\chapter{Discussion}
\label{chap:discuss}

\section{Method development}
\subsection{Haemocyte medium}
Even though the aggregation model was slightly over-dispersed, the estimates provide some insight into the three buffers relative abilities to prevent hemocyte aggregation within the first hour post-withdrawal. The combination of Ca$^{2+}$-free and \acrshort{edta}-containing buffers were effective inhibitors of hemocyte aggregation compared to simply diluting samples on ice. When the latter method was used, visible aggregates were usually formed within the syringes immediately after hemolymph aspiration - even though the \acrshort{mpss} was pre-chilled on ice. This observation shows that a two-fold dilution in MPSS is less than sufficient, such that a further dilution may be required for satisfactory effect. A many-fold dilution would be inconvenient in the preparation of haemolymph smears with a certain desired density, and too time-consuming for the acquisition of 10.000 events on a flow cytometer. This approach was therefore ruled out of question.

By comparing the inhibitory effects of \acrshort{mas} and \acrshort{acb} on haemocyte aggregation, our data suggests the slightly higher concentration of \acrshort{edta} in \acrshort{acb} compensates for the lack of citrate. Since high concentrations of \acrshort{edta} have been reported to impair haemocyte viability (\cite{Grandiosa2018, Burkhard2009}), a direct comparison of \acrshort{mas} and \acrshort{acb} with regards to acute effects on viability was required to identify the most suitable buffer of the two.

Our results suggest that \acrshort{edta} is cytotoxic to haemocytes at the concentrations used in both \acrshort{acb} (13.4 mM) and \acrshort{mas} (11.5 mM), since these buffers caused a significant dose-dependent increase in the percentage of necrotic haemocytes across the three timepoints (Table \ref{tb:Paired_ttests}). However, since there was no significant differences between the \acrshort{edta}-containing buffers and the negative control group after 15 minutes of incubation, this cytotoxic effect had no detectable manifestation within the time-frame of the planned flow cytometric assay. As long as haemolymph samples are stained and processed within 30 minutes of sampling, a flow cytometric dye exclusion/inclusion assay with TO-PRO$^{TM}$-3 Iodide and \acrshort{calceinam} will not have time to detect \emph{in vitro} necrosis caused by the buffers themselves.

The apoptosis assay with non-adjusted MAS (pH = 6.1) showed that 15 minutes was more than enough for haemocytes to enter programmed cell death. The percentage of cells already in late apoptosis suggests that an abrupt decrease in pH is an efficacious inducer of apoptosis, and that the haemocytes of \emph{M. edulis} are very sensitive to the environmental pH. This has been demonstrated by several recent studies investigating the effects of ocean acidification on the immune system of bivalves (see e.g., \cite{Wang2016, Dang2023}). Maintaining the pH of MAS at 6.1 is therefore not an option for flow cytometric analyses of apotosis or necrosis.

Our analyses did not detect any differences between ACB and MAS (pH = 7.0) with regard to their anticoagulant effects or cytotoxicity within the time-frame of the planned assays. The data does however demonstrate the importance of a carefully regulated haemocyte medium pH. As the buffer capacity of pH-adjusted MAS is (pH = 7.0) negligible, ACB appears as the most suitable choice of haemocyte medium. The Anticoagulant Buffer (ACB) of Pipe et al. (1997) was therefore used for flow cytometric analyses in the Hybrid MN Cytome Assay.
