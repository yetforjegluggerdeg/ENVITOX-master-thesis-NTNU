\chapter{Discussion}
\label{chap:discuss}

This is where you write about your results in the context of what was expected, what others with similar experiments have found (in line with, contradictory, similar to). Put it together with other findings that may be relevant or interesting. The results are interpreted: what do they mean for the field, the risk assessors, the environmental risk of Ti2O, ZnO Ag NPs in the aquatic environment, Trondheimsfjorden. Extrapolate to a greater level of organization?. Put the results into a bigger context.

A small proportion of large cells displayed a strangely similar morphology to the eosinophils, only distinguishable by the fact that their cytoplasm remained unstained with eosin. Whether these cells were a result of staining artifacts or incomplete staining, is not entirely known or easy to determine. Since they fit into the eosinophils according to size and complexity in cluster 3, we assumed them to be incompletely stained eosinophils, and counted them as such when encountered during MN scoring. Discuss this in light of Le Foll et al, 2010 findings, and the fact that we (almost) never encountered them when staining adhered cells according to Bolognesi's MN protocol. Most likely a result of incomplete staining in suspension.

We should have withdrawn two samples from each mussel (as we did), but one into MPSS for staining with the Hemacolor kit and subsequant MN scoring (cold MPSS, 5 min inc in humid chamber) and one into ACB for FCM assays. With a bit dilution in MPSS and short incubation, neither aggregation nor spreading is a problem.

Haemolymph smears can be messy. Therefore, scoring of necrotic and apoptotic haemocytes by microscopy should be performed by certified operators with formal education or practice from hematology or immunology. A few example PNG images with from a published protocol leaves too much to subjectivity. Secondly, if they are rare events (e.g., low dose), why not score all 10 of 100.000 cells objectively by FCM instead of 0 or 1 from 2000 cells in the smear?