\chapter{Material and method}
\label{chap:m&m}

\section{Material}

\subsection{Laboratory instruments}
BD Accuri C6 Plus benchtop flow cytometer (BD Nordics (prev. Puls Norway), Norway). Filters and lasers.
CytoSub submersible flow cytometer (CytoBuoy, Netherlands)
Coulter Counter Multisizer4 (Beckman Coulter, US) eqipped with a 100 \micro m aperture (size-range 2-60 \micro m)

Nikon Eclipse Ni-U
- Semrock Brightline GFP-4050B filter-cube
- Semrock Brightline LED-Cy5-A
- Plan Fluor 40x/0.75 water immersion objective,
- Plan Fluor 100x/1.30 Oil immersion objective

Nikon Eclipse 90i
- Filter specs?
- Nikon Plan Apo 60XA/1.40 Oil immersion objectice
- Nikon Plan Apo VC 100X/1.40 Oil immersion objective

Leitz Labrolux 12 binocular microscope (Leica Mikroskopi AS, Norway) [EF 40/0.65 objective]
Eppendorf Centrifuge 5804 R (Eppendorf, Norway), [rotor: A-4-44, rotor radius: 15.5 cm], high-speed refrigerated benchtop centrifuge
Jouan KR22i floor centrifuge (Thermo Fischer Scientific, US), high-speed high capacity refrigerated floor centrifuge 8 [rotor: AK 100-21] angle?
Bürker Counting Chamber (Hirschmann Laborgeräte, Germany) with 0.1 mm depth of chamber
Eirik Lund sitt kamera, lense og imaging software: 
Sony ILCE A6400 with E-mount, lense: Tamron 17-70mm F/2.8 
Adobe\textsuperscript{\textregistered} Lightroom Classic 12.0 

\begin{table}[H]
	\centering
	%\caption{Chemicals used in the master thesis, listed alphabetically according to chemical name, including the chemical's CAS nr., purity/grade, supplier and state.}
	\label{tb:instruments}
	\resizebox{\linewidth}{!}{
	\begin{tabular}{lll}
	\textbf{Instrument} & \textbf{Model} & \textbf{Producer} \\
		\midrule
   Benchtop Flow Cytometer               & BD Accuri$^{TM}$ C6 Plus & BD Biosciences \\
   Submersible Flow Cytometer            & Cytosub                  & CytoBuoy \\
   Coulter Counter                       & Multisizer 4             & Beckman Coulter \\
   Upright microscope                    & Eclipse Ni-U             & Nikon \\
   Upright microscope                    & Ecliplse 90i             & Nikon \\
   Transmitted/incident light microscope & Labrolux 12              & Leitz \\
   Light engine                          & Sola SM II 365           & Lumencor \\
   Light engine                          & EL6000                   & Leica Microsystems \\
    CMOS camera                           & MC170HD                  & Leica Microsystems \\
   Microscope camera                      & DS-Fi1                   & Nikon \\
   Altin's fluorescence camera           & ?                        & Nikon \\
   Microscope camera                     & 4KHDMI                   & DeltaPix \\
   Benchtop centrifuge                   & Centrifuge 5804 R        & Eppendorf\\
   Floor centrifuge                      & KR22i                    & Jouan \\
   Counting chamber                      & Bürker                   & Hirschmann \\
   		\bottomrule
	\end{tabular}
}
\end{table}



\subsection{Chemicals}
\begin{table}[H]
	\centering
	%\caption{Chemicals used in the master thesis, listed alphabetically according to chemical name, including the chemical's CAS nr., purity/grade, supplier and state.}
	\label{tb:chemical-list}
	\resizebox{\linewidth}{!}{
	\begin{tabular}{lllll}
	\textbf{Chemicals (abbrv.)} & \textbf{CAS-No.} & \textbf{Purity/grade} & \textbf{Supplier} & \textbf{state} \\
		\midrule
    Calcium chloride dihydrate      & 10035-04-8 & $\geq$ 99.0 \% & Sigma Aldrich & s \\
    Dimethyl sulfoxide              & 67-68-5    & $\geq$ 99.5 \% & Sigma Aldrich & l \\
    D-(+)-Glucose                   & 50-99-7    & $\geq$ 99.5    & Sigma Aldrich & s \\
    \ce{Na2EDTA}$\cdot$\ce{2H2O}    & 6381-92-6  & 98.5-101.5 \%  & Sigma Aldrich & s \\
    Ethanol                         & 64-17-5    & 96 \% vol      & VWR           & l \\
    \ce{Na2HPO4}$\cdot$\ce{2H2O}    & 10028-24-7 & $\geq$ 98.0    & Sigma Aldrich & s \\
    Potassium phosphate monobasic   & 7778-77-0  & $\geq$ 98.0    & Sigma Aldrich & s \\
    Copper(II)sulfate pentahydrate  & 7758-99-8  & $\geq$ 98.0    & Sigma Aldrich & s \\
    Formaldehyde                    & 50-00-0    & 37\% wt        & Sigma Aldrich & l \\
    HEPES                           & 7365-45-9  & $\geq$ 99.5 \% & Sigma Aldrich & s \\
    Magnesium sulfate heptahydrate  & 10034-99-8 & $\geq$ 99.5 \% & Sigma Aldrich & s \\
    Methanol                        & 67-56-1    & $\geq$ 99.9 \% & Sigma Aldrich & l \\
    Potassium chloride              & 7447-40-7  & $\geq$ 99.9 \% & Sigma Aldrich & s \\
    Sodium chloride                 & 7647-14-5  & $\geq$ 99.5 \% & Merck         & s \\
    TRIS(hydroxymethyl)aminomethane & 77-86-1    & ACS reagent    & Merck         & s \\
    TRIS HCl                        & 1185-53-1  & $\geq$ 99.0 \% & Sigma Aldrich & s \\
		\bottomrule
	\end{tabular}
	}
\end{table}


\subsection{Reagents for Flow Cytometry}
\begin{table}[H]
	\centering
	%\caption{Reagents and kits used in the master thesis, listed alphabetically according to product name, including manufacturer, supplier and supplier's catalogue number.}
	\label{tb:reagent-list}
	\resizebox{\linewidth}{!}{
	\begin{tabular}{lllll}
	\textbf{Product name (abbrv.)} & \textbf{Manufacturer} & \textbf{Supplier} & \textbf{Catalogue} & \textbf{Concentration} \\
		\midrule
    TO-PRO$^{TM}$-3 Iodide (642/661) &  InVitrogen$^{TM}$  & Thermo Fisher & T3605 & 1.2 \micro M \\
    Ethidium Homodimer-1 &  InVitrogen$^{TM}$ & Thermo Fisher &  E1169 & 4 \micro L/sample \\
    Apotracker$^{TM}$ Green & BioLegend & Fisher Scientific & 50-207-9934 & 560 nM \\
    Calcein-AM & Invitrogen$^{TM}$ & Thermo Fisher & C1430 & 170 nM \\ 
    CS\&T RUO beads & BD Biosciences & BD Biosciences & 661414 &  4 drops/mL \\
    8-peak validation beads & Spherotech & BD Biosciences & 653144 & 4 drops/mL \\
    6-peak validation beads & Spherotech & BD Biosciences & 653145 & 4 drops/mL \\
		\bottomrule
	\end{tabular}
	}
\end{table}


\subsection{Microscopy kits and reagents}
\begin{table}[H]
	\centering
	%\caption{Reagents and kits used in the master thesis, listed alphabetically according to product name, including manufacturer, supplier and supplier's catalogue number.}
	\label{tb:Microscopy-list}
	\resizebox{\linewidth}{!}{
	\begin{tabular}{llll}
	\textbf{Product name (abbrv.)} & \textbf{Producer} & \textbf{Supplier} & \textbf{Catalogue} \\
		\midrule
    Giemsa's azur eosin methylene blue solution & Merck & Sigma Aldrich & 1.09204.0500 \\
    Hemacolor\textsuperscript{\textregistered} & Sigma Aldrich & Sigma Aldrich & 1.11661 \\
    Eukitt\textsuperscript{\textregistered} Quick-hardening mounting medium & Orsatec GmbH & Sigma Aldrich & 03989 \\
    Type N Immersion Oil for Microscopy & Nikon & ? & MXA20234 \\
    Methanol & Merck & Sigma Aldrich & 1.06009.2511 \\
    Percoll$^{TM}$ & Cytiva Sweden AB & Sigma Aldrich & GE17-0891-02 \\
		\bottomrule
	\end{tabular}
	}
\end{table}



\subsection{Buffers and solutions}
\begin{table}[H]
	\centering
	\label{tb:buffers}
	\resizebox{\linewidth}{!}{
	\begin{tabular}{ll}
	\textbf{Buffer} & \textbf{Composition} \\
		\midrule
    MAS                   &  375.6 mM \ce{NaCl}, 28.97 mM Citric Acid$\cdot$3Na$\cdot$2\ce{H2O}, 113.8 mM D-Glucose, \\ 
                          & 2.617 mM Citric Acid$\cdot$\ce{H2O}, 11.5 mM \ce{Na2EDTA}$\cdot$\ce{2H2O}, pH=7.0 \\
    Anticoagulant buffer  & 55.5 mM D-glucose, 171.1 mM NaCl, 13.43 mM \ce{Na2EDTA}$\cdot$\ce{2H2O}, \\
                          & 0.05 M TRIS/HCl, pH=7.6 \\ 
    PBS                                  & 136.9 mM \ce{NaCl}, 2.7 mM \ce{KCl}, 10.1 mM \ce{Na2HPO4}, 1.8 mM \ce{KH2PO4} \\
    Sorensen Buffer       & 66.67 mM \ce{KH2PO4}, 66.67 mM \ce{Na2HPO4}$\cdot$\ce{2H2O}, pH=6.8 \\
    Hemolymph solution    &  470 mM \ce{NaCl}, 10 mM \ce{KCl}, 10 mM \ce{CaCl2}, 10 mM HEPES \\
                          & 47.7 mM \ce{MgSO4}, pH=7.41 \\
		\bottomrule
	\end{tabular}
	}
\end{table}

BD Microlance$^{TM}$ 3, 23G x 1" - Nr. 16, 0.6mm x 25 mm, sterile, REF 300800
HENKE-JECT 1 mL syringes, 1mL, luer, REF 8300014579, HENKE SASS WOLF GmbH, Tuttlingen, Germany

\section{Method}
\subsection{Animal housing}
Adult blue mussels (\emph{Mytilus edulis}) of x.x$\pm{5}$ cm shell length were obtained from Snadder og Snaskum AS (Indre Fosen, Norway). Upon arrival at the marine animal housing facilities of NTNU, Centre of Fisheries and Aquaculture (SeaLab), the mussels were transferred to 50 L filtered seawater flow-through tanks (11 L/min) supplied by a direct inlet from Trondheimsfjorden at 80 m depth ($\SI{7.5}{\celsius}$). Here, the mussels were kept to acclimatize for 2 days before transfer to the experimental exposure setup.

Time of mussel harvest/purchuse?

Describe water treatment in more detail: include information regarding sand-filter, protein-skimmer, 0.5 um filter bags and UV-treatment.

Because the watertreatment left no natural feed for the mussels, the filtered seawater were suplemented with algea (species) [frequency]


\subsection{Experimental setup/design}
- Specs related to flowrate, feeding (alge conc: Coulter counter), how often was nano particles changed (every 3-4 days: check article), concentrations (ICP-MS --> mg/L in stocks and exp tanks), zetasizer --> size and surface potential

include figure of exp setup?

depuration period: check article

storage from depuration to sampling: mussels were not kept on ice, but were washed and taken directly upstairs for measuring weight, length, height and width for condition index, and were given to us for FCM sampling afterwards. If we were delayed, they were kept in the fridge until hemolymph sampling 


\subsection{Hemolymph collection}
By employing the sampling technique described above, 0.5 mL hemolymph was withdrawn from each mussel into 1.0 mL syringes pre-filled with 0.5 mL anticoagulant buffer (ACB).

- Hemolymph was not pooled or centrifuged. 

\subsection{Determination of hemocyte concentration and subpopulations}
"The total hemocyte concentration, morphometry and definition of sub-populations were determined using side scatter (SSC) and forward scatter (FSC) light. Side scatter light corresponds to the relative internal complexity or granularity of the cells and FSC corresponds to the relative cell size" \cite{Rolton2020}


Concider including the experiment that was performed to validate the BD Accuri C6 Plus hemocyte counting technique with a technique using counting beads. Mention that a hemocytometer/counting chamber was used for the initial method development. It's a lot of data and work, which never looks bad. [Maybe not a separate subsection, but include under FCM or elsewere?]

\subsection{Flow Cytometry}
Flow Cytometer used and Flow Cytometry acquisition software
External software used for graphing and analysis of exported FCS files.
Replicate/triplicate measurements?
Number of events recorded for each mussel? ()
Briefly describe FCS and SSC.
Mention if data were collected in linear or logarithmic scale, 
FSC treshold (80.000 FSC-H)
Fluorescent compensation (matrix)
Describe gating strategy herein? Debris exclusion; doublet exclusion --> hemocytes. (Gating of basophils and eosinophils, the use of FMO controls with TO-PRO-3 and/or the apoptosis stain) to create gates.


\begin{table}[H]
	\centering
	\caption{The FCM acquisition and fluidics settings specified with the BD Accuri C6 Plus acquisition software during the flow cytometric experiments reported in this work.}
	\label{tb:FCM_settings}
	\resizebox{\linewidth}{!}{
	\begin{tabular}{lllll}
	\textbf{Experiment nr.} & \textbf{Event-triggering threshold} & \textbf{Acquisition stop-condition} & \textbf{Flow rate (\micro L/min)} & \textbf{Core size (\micro m)} \\
		\midrule
    Aggregation & 80.000 FSC-H & acquired volume, 20 \micro L & 30 & 10 \\
    Calcein AM and TO-PRO$^{TM}$-3 Iodide & 80.000 FSC-H & 10.000 hemocyte events & 36 & 16 \\
    Apo-15 and TO-PRO$^{TM}$-3 Iodide & 80.000 FSC-H & 10.000 hemocyte events & 36 & 16 \\
		\bottomrule
	\end{tabular}
	}
\end{table}

\subsection{Slide preparation}
Refer to the MN cytome assay by Bolognesi, and mention that we fixed and stained hemocytes in suspension to produce slides without aggregated hemocytes. Since the cells were dead, they did not spread or produce pseudopodia, which allowed us to observe the actual size and form which they would posses ass they flowed through the laser of the flow cytometer.

\subsection{Assay Validation}
Both dead cell stain and apoptosis stain has to be validated. For TO-PRO-3 perform the same experiment with 70\% methanol-killed cells as you did for Calcein-AM + EthD-1, and use Calcein-AM to stain live cells.
How to do this for the Apo-stain?

\subsection{Measurements and calculations}
(\cite{R-project})