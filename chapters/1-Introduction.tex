\chapter{Introduction}
\section{The mussel micronucleus cytome Assay}
Micronuclei (\acrshort{mni}) are small cytosolic membrane-enclosed chromatin bodies that contain acentric chromosome fragments or whole lagging chromosomes that remain outside the nucleus of daughter cells after cell division (\cite{Fenech2011}). Micronuclei with acentric fragments can originate from unrepaired or misrepaired DNA-breaks from interactions with clastogenic chemicals, while larger MNi with whole chromosomes arise from indirect interactions with the replication apparatus during anaphase (aneugenic mechanism) (\cite{Fenech2011}). While the two structures may provide mechanistic information about the tested chemical, they are not readily distinguishable in standard cytologic preparations (\cite{Natarajan1993}). Without specific labeling of kinetochores or centromere-specific DNA, MNi provide general evidence for accumulated direct or indirect genotoxic damage during the cell life (\cite{Tucker1996, Lynch1993}).

These cytogenic damages are the major endpoints of micronucleus assays, which represent instrumental tests in the hazard identification of genotoxic compounds (\cite{OECD474, OECD487, USEPA1998}). Since there is a strong association between specific cytogenic alterations and tumorigenesis, the implementation of this biomarker in toxicological risk assessment is well justified (\cite{Mitelman1983} in: \cite{Tucker1996}). Micronucleus tests are performed on dividing or newly divided cells, and are most typically used to assay genotoxic damage in cells from bone marrow samples and blood in the case of mammalian test systems (\cite{Heddle1983, Warheit2018b}).

Micronucleus tests have also been extended for use in ecotoxicology, and represent one of the most prevalent biomarkers of genotoxicity in aquatic animals (\cite{Bolognesi2012, Bolognesi2014}). By following the cytome approach applied i mammalian systems (\cite{Fenech2007}), Bolognesi and Fenech (2012) further updated and refined the existing MN test for bivalve haemocytes and gill cells to include scoring of necrotic and apoptotic cells as endpoints for cytotoxicity (\emph{The MUssel MN Cytome assay}, MUMNcyt). Scoring of cells with nuclear buds (\acrshort{nbuds}) were also included as a biomarker of elimination of amplified DNA and/or DNA repair complexes, although the mechanism leading to NBUD formation is not completely known (\cite{Bolognesi2012}). Since the \acrshort{mumncyt} assay also incorporates a differential haemocyte count, this assay also opens for....

Bivalves have gained a central role as sentinels in marine ecotoxicology studies, and particularly the widespread blue mussel (\emph{Mytilus spp.}), which has become the focal point of national monitoring programs of coastal water pollution (\cite{Goldberg1975, Beyer2017b, Cajarville2000}). Because of their limited biotransformation capacity (\cite{Beyer2017b}), low mortality (\cite{Ale2019}) and ability to accumulate a wide range of pollutants as filter-feeders; this marine invertebrate represents a suitable system for monitoring and assessing the impacts of anthropogenic pollution on the health of marine coastal ecosystems. Furthermore, they... [Concider bridging with their key ecological, commercial importance and suceptibility to bachteria - which has caused population declines in France, Spain with examples: "It is possible that changes in circulating cell types may indicate stress and provide information on immunological status (Couch, J.A. 1992 - Pathobiology of Marine and Estuarine Organisms)"]

Next paragraph: The MUMNcyt assay represents a well "bla bla" assay, capable of uncovering potential immunomodulating effects of toxicants, as well as the underlying mechanisms of this shifts (apoptosis/necrosis). But

[This part is potentially removed if the above bridge is chosen] Their feeding strategy in particular make them highly effective in micro- and nano-scaled particle uptake (\cite{Canesi2012}), which has led to their use in a large pool of studies on the effects of engineered nanoparticles (\cite{Rocha2015}). These studies often integrate a battery of cellular and molecular biomarkers of pollutant-induced stress and immunomodulation, where the \acrshort{mumncyt} assay is included to evaluate potential cytogenotoxic effects (see e.g., \cite{Rocha2014, Ruiz2015}).


Unlike oysters and clams, which also represent bivalve molluscs of major ecological and economical interest, mussels areactually not subjected to episodes of massive mortality (\cite{Costa2009} in: \cite{Rioult2014}).

Mussel hemocytes represent a valuable model of invertebrate innate immune cells, in which modulatory effects of xenobiotics (onto ABC transporters) can be assessed. Formulation from \cite{Rioult2014}.


\newpage

It is a suspension feeder, filtering high quantities of water for suspended particulate matter (\cite{Beyer2017b}). This feeding strategy makes them highly effective in micro- and nano-scaled particle uptake, and consequently especially susceptible to engineered NP exposure (\cite{Canesi2012}).




Important commercial species and a KEY component of marine coastal ecosystems: \emph{Mytilus spp.}: blue mussels are ecologically important as they provide essential ecological services such as food and habitat to a multitude of other species. In: Beyer et al., 2017

Sessile lifestyle
Distribution near in coastal areas/shores near localities of human activities
Limited biotransformation capacity (\cite{Beyer2017b}) + low mortalitiy/loss (\cite{Ale2019})
Filter-feeding - effective in NP uptake (\cite{Canesi2012})



General description of their distribution, sessile lifestyle, ability to accumulate pollutants without high mortality, feeding strategy etc. The common blue mussel (\emph{Mytilus edulis}) is a marine bivalve residing on hard substrates of the sediment/water interphase and intertidal zones of temperate waters.  \emph{Mytilus sp.} has served as sentinel species in marine monitoring programs since the 1970s (\cite{Goldberg1975}), are highly suitable for assessing engineered NP toxicity. Bridge that leads on to their immune system and use as systems of immunotoxicity and genotoxicity. See \cite{Rocha2015} for information of on what species are most commonly used in ecotox studies.

\emph{Mytilus sp.} 53.4\% of papers related to ecotoxicity of ENMs until december 2014 (\cite{Rocha2015}).


\section{Bivalve hemocytes as \emph{in vivo} model systems for immunotoxicity and genotoxicity}
Used as membrane integrity model system. "Since their membranes are susceptible to being destabilized by different stressors, this feature has been frequently used as a biomarker to monitor pollution and animal health (reviewed in Moore et al. 2004, 2006)." From "Changes induced by two strains of Vibrio splendidus in haemocyte subpopulations of Mya arenaria, detected by flow cytometry with LysoTracker" (Mateo, 2009), DOI: 10.3354/dao02121. The total hemocyte count (\acrshort{thc}) decreased by 66\% after a bacterial injection \cite{Parisi2008}. THC reference: \cite{Ciacci2012}. THC in mussels at oil spill cite increased as with time from accident (\cite{Dyrynda1997}).

\subsubsection{Differential count: granular and agranular haemocytes per 1,000} %cytometric marameters = THC and DCC
In addition to scoring cytogenic damage (MNi and NBUDs) and cytotoxic alterations (necrosis/apoptosis), the protocol includes a differential count of agranular and granular haemocytes (per 1,000).

Just a theory: Given that the blast-like cells are the precursor of both granular cell types, and chemical insult leads to increased proliferation of blast-like cells - which increases the risk of neoplastic cells (prolif rate pos. corr. to mutations). Increased recruitment/hamatopoiesis/cell turnover rate

- Seasonal variation in agranular and granular (\%) haemocytes in M. galloprovincialis (\cite{Santarem1994}). May be related to reproduction/spawning (allocation of resouses/nutrients?)

READ LIVINGSTONE et al, 2000: immune response


Cd:
- Decrease in eosinophilic granulocytes in response to Cd QDs, while basophils increased in response to Cd2+. (Basopils incl. blasts). Suggested limitation of QD transport by EO, since they decreases (\cite{Rocha2014}). May offer insights into the target/susceptible cell-types in NP toxicology, which may indicate possible immunomodulating effects.
- No sig. change in proportions of haemocyte types (baso/eo) with Cd exposure, while total haemocyte count nearly doubled at 
  high exposure, and decreased relative to control in the low exposure group. Did not differentiate between blast-like and granular basophils (\cite{Coles1995}). % M. edulis
- Increased numbers of small hyaline (blast-like) cells within the hyalinocyte sub-population of C. gigas, following exposure 
  to 0.3 ppm cadmium ions, with a concomitant decrease in large cells (Auffret \& Oubella, 1994). THC increased with time of exposure relative to control (2x increase). % C. gigas
- THC increased following exposure to Vibrio tubiashi after preexposure with Cd (\cite{Pipe1995}) % M. edulis



Cu2+  
- Cu2+ + Vibrio tubiashi led to sig. decrease in eosinophils compared to basophils. No such effect was observed with Cd or fluoranthene (\cite{Pipe1995}) % M. edulis

  PAHs
- Basophilic haemocytes freq. showed sig. negative correlation with PAH concentrations after oil spill, while EO showed a sig. 
  positive correlation. Inverse corr. between 3 PAHs and THC. Immune parameters decreased, while EO increased: possible compensatory effect? No time-related effects (\cite{Dyrynda1997}). Small hyaline cells were positively correlated with total PAHs conc. mm. (never neg corr.) % M. edulis
- Dyrynda (1998) THC and DCC (Eo \%). No sig. diff beteen contaminated site and ctrl, but slight deacrease in both. Argues 
  that immunomodelation did not arise from THC or DCC changes. Observed in laboratory settings with sudden short-term exposure, but may adjust to longterm exposures.  (\cite{Dyrynda1998}) Did not score blast-like cells % M. edulis 


- A range of specific contaminants have been shown to induce alterations in the haemocyte profile of several bivalve species 
   (Anderson, 1993). review.

- Renwartz L. Variations in hemocyte counts in the mussel, Mytilus edulis: Similar reaction patterns occur in disappearance and return of molluscan hemocytes and vertebrate leukocytes. (2013)



Many authors reporting MN frequencies in \emph{Mytilus sp.} either fail to comment on the scored cell type (\cite{Ale2019}) or incorrectly attach photomicropgraphs of eosinophilic granulocytes labelled "agranular haemocyte" (\cite{Meng2020}).


Haemolymph smears can be messy. Spread haemocytes takes on a variety of shapes that are hard to recognize from one JPEG image.
Slightly different staining protocols might change the haemocyte morphologies drastically. E.g., hyalinocytes with Hemacolor staining, no such cell types were observed with the protocol by Bolognesi and Fenech (2012). Not all cells that Bolognesi and Fenech presented as agranular haemocytes (3-4 \micro m) in Fig. f and i matches their own description. The nuclei itself of these cells may be 5 um (\cite{Carballal1997}, p. 130).


Differential haemocyte counts should be performed by flow cytometry to avoid operator-subjective interpretations/classifications of the different haemocyte cells types. Choice of haemocyte medium (EDTA), method of fixation (and post-fixation), the cytological stain used (Giemsa, Hemacolor), pH of cytological staining solutions and slide preparation have an immense effect on the overall quality of of the preparations, and may produce strikingly different staining results and consequently variable interpretation of the observed haemocyte cell types. 

Manual scoring of cytogenic damage in defined cell types is a time-consuming and labor-intensive process with low throughput. When morphological cell alterations following necrotic and apoptotic cell death are

Bridge to semi-automation: inter-operator variability, subjectivity etc.




\section{Classification of the haemocyte subpopulations of \emph{M. edulis}}
\label{subsection:haemocyte_classification}
Since the first written account on the subject (\cite{Cuenot1891}, cited in: \cite{Cheng1980}), several authors have devoted their attentions to developing a unifying classification system for the amoebocytic blood cells of bivalve mollusks, more commonly known as haemocytes (\cite{Cheng1980, delaBallina2022}). Belonging to the bivalve familiy \emph{Mytilidae}, the haemocytes of \emph{Mytilus edulis}, \emph{Mytilus galloprovincialis} and several other commercially important species of the genus \emph{Mytilus} have been encompassed by these efforts, creating a substantial pool of literature on the haemocytes of this genus alone. Despite a lack of consensus for any unifying classification system for the haemocytes of this phylum at large, the literature that exists on the haemocytes of \emph{M. edulis} generally agrees on the existence of three distinct subpopulations.

The first effort to classify the haemocytes of \emph{M. edulis} was made by Moore and Lowe (1977). Much like the other attempts to classify bivalve haemocytes at the time, this classification was based on the morphofunctional aspects of these cells - a system that has been extensively reviewed by Hine (1999). Moore and Lowe constructed a simple classification based on static morphological and ultrastructural characteristics of the haemocytes, combined with their phagocytic capacities (\cite{Moore1977}). From routine cytological staining, they identified three haemocyte subpopulations (or cell types): "(1) small basophilic hyaline cells or lymphocytes, (2) larger basophilic hemocytes with varying degrees of irregular cytoplasmic granulation and vacuolation, and (3) eosinophilic granular haemocytes or granulocytes" (\cite{Moore1977}). The small basophilic cells (4-6 \micro m) were generally spherical in outline, had a scant thin rim of basophilic hyaline (read: transparent) cytoplasm and a spherical nucleus - bearing resemblance to vertebrate lymphocytes. The larger granular basophils (7-10 \micro m) displayed less intense basophilic cytoplasm, lower nuclear:cytoplasmic (N:C) ratios and more irregularly shaped nuclei. The eosinophilic granulocytes were the largest cell type identified (7-12 \micro m). They had a regular spherical appearance, further characterized by a small round nucleus, low N:C ratio, and a cytoplasm filled with spherical eosinophilic granules (0.5-1.0 \micro m).

Electron micrographs confirmed the existence of three ultrastructurally distinct morphologies. Except for a few mitochondria, the lymphocyte-like cells contained a scarcity of organelles and granules. This stood in sharp contrast to the larger granular basophils, which contained Golgi apparatus, phagosomes and smaller granular inclusions - possibly representing primary lysosomes. A phagocytosis assay with experimentally injected carbon particles revealed that both granular cell types displayed phagocytic properties, while the small lymphocyte-like cells did not show any evidence for this capacity (\cite{Moore1977}).

The morphological and ultrastructural findings of Moore and Lowe (1977) have since been confirmed by several investigators (\cite{Rasmussen1985, Renwartz1990, Pipe1990, Noel1994, Pipe1997, Wootton2003}). From their stand-alone electron microscopical examinations, Pipe and colleagues (1990) made a distinction between granular haemocytes with small (0.2-0.3 \micro m) and large (0.5-1.5 \micro m) granules. By relating the two ultrastructural phenotypes to their cytological staining properties, investigators soon demonstrated that the two cell types corresponded to the basophilic and eosinophilic granular haemocytes of Moore and Lowe (\cite{Pipe1990, Noel1994}). Thus, if reduced to it's static morphological criteria, Moore and Lowe's classification of \emph{M. edulis} haemocytes coincides with the original system of Cúenot (1891). This system generally recognized three types of haemocytes in bivalves: "(1) finely granular haemocytes, (2) coarsely granular haemocytes and (3) cells with very little cytoplasm surrounding the nucleus" (\cite{Cheng1984}). 

Leaning towards a phylum-wide two-categorical classification (hyalinocytes and granulocytes), Cheng (1981) argued that a distinction between the basophillic and eosinophilic granulocytes of \emph{M. edulis} was artificial, as he saw them as being immature and mature stages of the same cell type (granulocytes), respectively. From observations of what resembled intermediate stages between the lymphocyte-like and larger basophilic cells, Moore and Lowe (1977) had argued that the basophilic cells constituted an ontogenic developmental series, with the larger phagocytic macrophages representing the final stage of maturation. This was further supported by observations of lymphocyte-like cells with mitotic figures, suggesting that it could be the stem cell of this lineage (\cite{Moore1977}). Since a few smaller eosinophilic granulocytes (5-7 \micro m) were observed in their sections, the eosinophilic granulocytes were believed to represent a distinct growth series.

This theory, as pointed out by Cheng (1984), was primarily formulated through interpretive evaluations of morphological findings, rather than being based on direct ontogenic evidence. The classification of bivalve haemocytes should ideally be constructed on the basis of their ontogeny. However, mapping of ontogenic lineages among bivalve haemocytes have been tempered by the lack availible molecular databases, no one unifying model species, combined with uncertainty regarding the hematompoietic tissue(s) and processes of bivalves (\cite{Hine1999, Smith2016, Pila2016, delaBallina2022}). With no real ontogenic evidence to work with, a careful assessment of availible morphological data may represent a better alternative relative to a classification based solely on biochemistry and function (\cite{Hine1999}). 

\subsection{Flow cytometric classification of \emph{M. edulis} haemocytes}
Almost two decades after flow cytometers became commercially availible in the 1970s (\cite{Shapiro2004}), the application of these instruments started to gain traction within the field of invertebrate immunopathology (\cite{Fisher1988}). Since the traditional characterization of bivalve haemocytes were largely based on morphological criteria such as size, granularity and staining affinities, the simultaneous measurement of forward scatter (\acrshort{fsc}, $\approx$ size) and side scatter (\acrshort{ssc}, internal complexity $\approx$ granularity) represented a far less subjective approach to their characterization (\cite{AshtonAlcox1998, Allam2002, Mateo2009}).

A detailed flow cytometric characterization of the haemocytes of \emph{M. edulis} was undertaken by Le Foll and colleagues (2010), who were able to distinguish three subpopulations according to their cell diameters (\micro m) and Side Scatter (\acrshort{ssc}) (\cite{LeFoll2010}). These comprised one population of small cells (7.14$\pm{0.05}$ \micro m) with low \acrshort{ssc}, one population of larger cells (9.97$\pm{0.17}$ \micro m) with intermediate \acrshort{ssc} and one population of large cells (10.08$\pm{0.24}$ \micro m) with high \acrshort{ssc}. By running haemocytes stained with eosin - which is fluorescent in the green/yellow spectrum under blue laser exitation (\cite{Elfer2016, Koegle2020}) - their results suggested that the latter cluster to corresponded to the eosinophilic granulocytes. However, these results were not visually verified by microscopy. The use of flow cytometers equipped with cell sorting capabilities simplifies the process of verifying any classification derived from flow cytometric measurements (\cite{Shapiro2004}). However, when extracting cells with known measured characteristics is not possible, the cells to be classified can be separated by other means prior to flow cytometric acquisition.

Since the three haemocyte cell types of \emph{M. edulis} differ with regard to the size and density of their granules, Friebel (1995) and Pipe (1997) managed to physically separate the eosinophilic granulocytes from the two basophilic cell types by isopycnic centrifugation (\cite{Friebel1995, Pipe1997}). Depending on the fixative used, the whole haemocyte population separated into three or four distinct cell-bands in the interfaces of the various layers. The two basophillic cell types could be isolated in high purity from the upper cell band (lowest density), the eosinophilic granulocytes from the lower, while the intermediate fractions often consisted of varying proportions of all three cell types. Accompanied by the rapid growth of flow cytometric applications in invertebrate immunology, the progress made by Friebel (1995) and Pipe (1997) meant that results from functional and biochemical assays could be assigned to specific cell types at a higher throughput. Thus, the unraveling of the roles of individual haemocyte subpopulations really started to pick up speed.

\subsection{The role of haemocytes}
Present the difference in phagocytic capacity between the three cell types (include different types of materials).
Cytochemistry, enzyme content and cytotoxic secreted molecules in host defence, Surface receptors.
Ultrastructural findings: what organelles are found in the different cell types, and what they suggest about function.
The  eosinophilic granular cells in  M. edulis are the most active in phagocytosis and superoxide production (Pipe  et al., 1997)




























\section{Cellular methods}
Short introduction to the concept of measuring cell viability, and membrane integrity and apoptosis by flow cytometry. Leads to "in this assay, necrosis and apoptosis was measure by calcein A/ToPro3 and Apo15/ToPro3".

\subsection{Calcein acetoxymethyl ester}
Calcein acetoxymethyl ester (\acrshort{calceinam}) is among the most prevalent esterase-substrates used in viability assays today (\cite{Ramirez2010}). Similar to other electrically neutral fluorescein-derivatives, this small non-fluorescent molecule (994.9 g/mol) readily enters living cells through diffusion; where it is effectively trapped by intact cell membranes upon ester hydrolysis by intra-cellular esterases (\cite{Kaneshiro1993}). The retained poly-anionic calcein molecule bears five negative and two positive charges at pH 7, and is highly fluorescent in the visible spectrum (\acrshort{exmax}/\acrshort{emmax}: 495/515 nm) (\cite{Wallach1959}, in: \cite{Chiu1977}). Due to it's  insensitivity to pH in the physiological range, high photo-stability and superior cellular retention (\cite{Chiu1977, Kaneshiro1993}), calcein has several advantages over other esterase substrates commonly employed in viability assays (\cite{Ramirez2010}).

One critical drawback of employing calcein in standardized \emph{in vitro} viability/cytotoxicity assays resides in the fact that the acetoxymethyl tetraester form is a substrate of mammalian P-glycoprotein (P-gp) efflux pumps (\cite{Liminga1995}). Since different cell lines vary with regard to P-gp activity and hence the rate of calcein accumulation, results may not be directly comparable across different cell lines (\cite{Ramirez2010}). On the other hand, this exact property has led to the utilization of \acrshort{calceinam} as a fluorescent probe for assaying P-gp inhibitors (\cite{Di2016}). 
The inhibition of P-gp pumping activity by P-gp-inhibitors results in an increased accumulation of fluorescent calcein within the cells, which can then be detected and quantified through it's fluorescence (\cite{Tiberghien1996, Köhler2003}).

ATP-dependent efflux pump transcripts analogous to the mammalian P-gp and Multidrug Resistance-related protein (MRP) have also been detected and cloned from various bivalve tissues (\cite{Luckenbach2008, Luedeking2005}), including the haemocytes of \emph{M. edulis} (\cite{Rioult2014}). By using Calcein-AM as a fluorescent probe in combination with P-gp- and MRP-specific inhibitors, Rioult et al., (2014) demonstrated that the calcein-AM efflux activity in haemocytes of \emph{M. edulis} were carried out by a MRP-type transporter exclusively, and that this activity was higher and more inducible in eosinophilic granulocytes compared to the two other cell types. While MRP-induction assayed by calcein fluorescence may present a potential biomarker for marine biomonitoring studies (\cite{Rioult2014, Minier1998}), the differential accumulation of calcein in \emph{M. edulis} haemocytes also presents a potential novel probe for performing flow cytometric differential haemocyte counts in the common blue mussel, in conjunction with side-scatter measurements.

\subsection{Apo15/TO-PRO-3 Iodide apoptosis assay}
Theory behind Annexin-V/Apo-15 (\cite{Barth2020}) Annexin V has affinity for phosphatidylserine, which is externalized to the outer layer of the plasma membrane in the earlier stages of apoptosis. Annexin V also binds internal phosphatidylserine (\acrshort{ps}) in permeable membranes, i.e. dead cells. Thus, dead cells are Apo-15+ ToPro3+, while the early apoptotic cells are only Apo15+. The Annexin V/PI assay is considered the the gold standard of \emph{in vitro} apoptosis detection, but requires free Ca$^{2+}$.


\section{Objective}
