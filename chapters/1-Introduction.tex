\chapter{Introduction}
\section{The mussel micronucleus cytome Assay}
Genetic toxicology is the study of substances that can cause damage to the genetic material of an organism (mutagens and genotoxicants), potentially leading to mutations, chromosomal alterations and other harmful genetic effects. Allthough gene mutations form the genetic variation required for evolutionary adaptation (\cite{Harr2005}), non-neutral changes to the genetic material more frequently result in detrimental effects to individuals and their offspring (\cite{Halligan2006}). Genotoxic damage to somatic cells can contribute in cancer development (\cite{Hanahan2011}), while damage to germ cells can produce heritable disorders in offspring. Since these endpoints represent long-term effects from chemical exposure, regulatory agencies must rely on early warning responses (biomarkers) from \emph{in vivo} and \emph{in vitro} bioassays for their risk assessments (\cite{ECHA2017}). While the EU directive's mutagenicity testing is mainly focused on \emph{in vivo} germ cell tests, genotoxicity testing of new substances serves an important role in the initial screening for potential carcinogenic effects (\cite{Broschinski1998} in: \cite{OSPAR2020}) and for deciphering their mode of action (\acrshort{moa}) (\cite{Eastmond2009}). In this context, the induction of micronuclei (\acrshort{mni}) represents a core biomarker of genotoxic effects in somatic test systems.

Micronuclei are small cytosolic membrane-enclosed chromatin bodies containing acentric chromosome fragments or whole lagging chromosomes, that remain outside the nucleus of daughter cells after cell division (\cite{Fenech2011}). \acrshort{mni} with acentric fragments can originate from unrepaired or misrepaired DNA-breaks from interactions with clastogenic chemicals, while MNi containing whole chromosomes arise from indirect interactions of aneugenic chemicals with the replication apparatus during anaphase (\cite{Fenech2011}). While the two structures may provide mechanistic information about the tested chemical, they are not readily distinguishable in standard cytological preparations (\cite{Natarajan1993}). Without specific labeling of kinetochores or centromere-specific DNA, MNi provide general evidence for accumulated direct or indirect genotoxic damage during the cell life (\cite{Tucker1996, Lynch1993}).

These cytogenic damages are the major endpoints of micronucleus tests, which represent instrumental assays in the risk characterization of genotoxic compounds (\cite{UNEP2006, ICES2012, OECD474, OECD487, USEPA1998}). Since there is a strong association between specific cytogenic alterations and tumorigenesis, the implementation of MN as a biomarker in toxicological risk assessment is well justified (\cite{Zhang2015}, \cite{Mitelman1983} in: \cite{Tucker1996}). Micronucleus tests are performed on dividing or newly divided cells, and are most typically used to assay genotoxic damage in cells from bone marrow samples and blood in the case of mammalian test systems (\cite{Heddle1983, Warheit2018b}). Since the marine environment represents the ultimate recipient and sink of toxicants from land-based sources (\cite{ICES2012}), the deleterious effects of genotoxic compounds are not restricted to life on land.

Micronucleus tests were more recently extended for use in ecotoxicology, and represent one of the most prevalent biomarkers of genotoxicity in aquatic animals (\cite{Bolognesi2011}). To date, such eco-genotoxicity assays have most notably been performed in fish erythrocytes (reviewed in \cite{Agostini2021}) and in bivalve haemocytes (blood cells) and gill cells (reviewed in \cite{Bolognesi2014}). Bivalve mussels have gained a central role as sentinels in marine ecotoxicology studies, and the widespread blue mussel (\emph{Mytilus spp.}) has received a lot of attention in particular. Because of their sedentary lifestyle, limited biotransformation capacity (\cite{Beyer2017b}), low mortality (\cite{Ale2019, Costa2009}) and ability to accumulate a wide range of pollutants as filter-feeders (\cite{Viarengo1991}); this hardy invertebrate has become the focal point of national monitoring programs for coastal pollution in more than 50 coastal nations (\cite{Cantillo1998} in: \cite{Beyer2017b}), including the Norwegian \emph{Contaminants in coastal waters} program (Miljøgifter i kystområdene - MILKYS). In international marine pollution programs, micronucleus tests on caged mussels have been proposed as the only core biomarker of genotoxicity (\cite{Bolognesi2012}).

By following the cytome approach applied in mammalian systems (\cite{Fenech2007}), Bolognesi and Fenech (2012) updated and refined the existing MN test for bivalve haemocytes and gill cells to include scoring of necrotic and apoptotic cells as endpoints of cytotoxicity. Nuclear buds (\acrshort{nbuds}) were also included as a biomarker of genotoxicity, although the mechanism leading to NBUD formation is not completely known (reviewed in \cite{Fenech2011}). NBUDs are nuclear anomalies that are characterized by their morphological resemblance to MNi, except that they are connected to the main nucleus by a stalk of nucleoplasmic material (\cite{Fenech2002}). In addition to scoring cytogenic damage (MNi and NBUDs) and cytotoxic alterations (necrosis and apoptosis), a differential count of the agranular and granular haemocytes cell types were included in the protocol presented by Bolognesi and Fenech. As these innate immune cells have been found to exhibit different immunocompetences with regard to phagocytic activity, encapsulation and the secretion of cytotoxic molecules in host defence (\cite{delaBallina2022}), changes in the circulating haemocyte cell types may possibly serve as indicators of stress and provide information on the immunological health status of mussels (\cite{Couch1993}).

Several contaminants have been shown to induce changes in the haemocyte profile of bivalves (\cite{Couch1993}). Among the tested contaminants, these include Cd$^{2+}$ and Cd-based quantum dots (\cite{Rocha2014, Auffret1994}), Cu$^{2+}$ (\cite{Pipe1995, Pipe1999}), phenol (\cite{Fries1980}) and certain \acrshort{pahs} (\cite{Dyrynda1998, Dyrynda2000}). Since the mussel micronucleus cytome assay includes a differential haemocyte count in combination with two biomarkers of cytotoxic damage, an integrated interpretation of these endpoints might provide mechanistic information about the immuno-modulating effects of cytotoxic chemicals. Moreover, as increased hematopoiesis or regenerative hyperplasia from cytotoxic events increase the risk of neoplastic developments (\cite{Eastmond2012}), the differential haemocyte count also has potential to indicate a non-receptor mediated \acrshort{moa} for substances that are not DNA-reactive. This possibility requires a differential haemocyte count that discriminates immature "pro-haemocytes" from mature cell types.

In spite the widespread adoption of the micronucleus cytome assay, this protocol has several limitations or drawbacks; most of which has the potential to be addressed by flow cytometry. First of all, the entire procedure is reported to take approximately 1 hour per individual mussel (\cite{Bolognesi2012}). If multiple sampling sites are employed with 10 individuals/site, the time required for a larger study can reach substantial proportions. Secondly, the added complexity of scoring cytotoxic alterations, genotoxic damages and performing differential counts in parallel might divert attention from the main endpoint. As MNi are rare events scored among a relatively small subset of cells (> 1000 agranular haemocytes), a few false negatives can potentially have a significant effect on the MNi frequencies of mussels from sites with low contaminant levels. The third limitation is related to the complexity of the haemocyte cell types, and the operator's subjective interpretations or classification of these and their cytotoxic alterations in the haemocyte preparations. 

The classification of bivalve haemocytes is a comprehensive literature that is dominated by a lack of scientific consensus (\cite{Hine1999}). For operators that rely on the images provided by the protocol for their practical classification, the first inspection of a haemolymph preparation may not paint a picture that is all that intuitive. Bolognesi and Fenech (2012) proposed scoring of MNi in agranular haemocytes, which were characterized by their small size (3-4 \micro m), high nuclear:cytoplasmic (N:C) ratio and a lack or low abundance of cytoplasmic granules and organelles. As slight deviations from the staining protocol might produce striking differences in the overall staining result, the subjective interpretations of the observed cell types may produce large inter-operator variability in the scored cell population and the differential haemocyte count. This is highlighted by the fact that most operators fail to comment on the scored cell type (\cite{Bolognesi2012}), or erroneously score what they believe to be agranular haemocytes (e.g., \cite{Meng2020}). Since the granular haemocytes of \emph{M. galloprovincialis} are reportedly less sensitive to MN induction (\cite{Venier1997}), confusion of this sort can possibly reduce the sensitivity and inter-lab comparability of the micronucleus cytome assay. 

The difficulty of producing high quality preparations of haemolymph smears is another factor that might further complicate this issue. Haemolymph smears often have numerous staining artifacts and cells that are damaged during preparation. As a consequence, distinguishing cytotoxic alterations from staining artifacts may be challenging to operators without comprehensive training in histology or hematology. The identification of necrotic haemocytes based on extensive cytoplasmic vacuolation might be especially challenging in this context, as both granular cell types of \emph{M. edulis} are phagocytic and often display numerous phagosomes in their cytoplasm (\cite{Moore1977}). Damaged cell membranes - and the paler cytoplasmic staining that results from it - may also be hard to distinguish from irrelevant staining artifacts. 

In short, manual scoring of cytogenic damage in a defined population of bivalve haemocytes is a time-consuming and labor-intensive process with low throughput. The expanded scope of the micronucleus cytome assay further aggravates this fact, and a reference material consisting of a few PNG images leaves too much room for subjectivity in the identification of haemocyte cell types and cytotoxic alterations. As flow cytometers are permanent fixtures of contemporary cell biology labs, this high throughput cytologic instrument is availible to most ecotoxicologists involved in cellular work. With the capacity to differentiate single cells based on size, granularity and molecular markers, flow cytometry is a powerful tool for performing differential blood cells counts (\cite{Shapiro2004}), as well as scoring necrotic and apoptotic cells (\cite{Shapiro2003}). A hybrid flow cytometry/microscopy protocol of the \emph{Mussel micronucleus cytome assay} has therefore great potential to improve the power and efficiency of the existing methodology, while creating less room for subjectivity.

\section{Classification of haemocyte subpopulations in \emph{M. edulis}}
\label{subsection:haemocyte_classification}
Since the first written account on the subject of bivalve haemocyte classification (\cite{Cuenot1891}, in: \cite{Cheng1980}), several authors have devoted their attention to develop a unifying classification system for the amoebocytic blood cells of bivalve molluscs, more commonly known as haemocytes (\cite{Cheng1980, delaBallina2022}). The haemocytes of \emph{Mytilus edulis}, \emph{Mytilus galloprovincialis} and other commercially important species of the genus \emph{Mytilus} have been encompassed by these efforts, which has created a substantial pool of literature on the haemocytes of this genus alone. Despite a lack of consensus for any unifying classification system for the haemocytes of this phylum at large, the literature that exists on the haemocytes of \emph{M. edulis} generally agrees on the existence of three distinct subpopulations.

The first effort to classify the haemocytes of \emph{M. edulis} was made by Moore and Lowe (1977). Much like other attempts to classify bivalve haemocytes at the time, this classification was based on the morphofunctional aspects of these cells - a system that has been extensively reviewed by Hine (1999). Moore and Lowe constructed a simple classification based on static morphological and ultrastructural characteristics of the haemocytes, combined with their phagocytic capacities (\cite{Moore1977}). From routine cytological staining, they identified three haemocyte subpopulations (or cell types): "(1) small basophilic hyaline cells or lymphocytes, (2) larger basophilic hemocytes with varying degrees of irregular cytoplasmic granulation and vacuolation, and (3) eosinophilic granular haemocytes or granulocytes" (\cite{Moore1977}). The small basophilic cells (4-6 \micro m) were generally spherical in outline, had a scant thin rim of basophilic hyaline (read: transparent) cytoplasm and a spherical nucleus - bearing resemblance to vertebrate lymphocytes. The larger granular basophils (7-10 \micro m) displayed less intense basophilic cytoplasm, lower nuclear:cytoplasmic (N:C) ratios and more irregularly shaped nuclei. The eosinophilic granulocytes were the largest cell type identified (7-12 \micro m). They had a regular spherical appearance, further characterized by a small round nucleus, low N:C ratio and a cytoplasm filled with spherical eosinophilic granules (0.5-1.0 \micro m).

Electron micrographs confirmed the existence of three ultrastructurally distinct morphologies. Except for a few mitochondria, the lymphocyte-like cells contained a scarcity of organelles and granules. This stood in sharp contrast to the larger granular basophils, which contained Golgi apparatus, phagosomes and smaller granular inclusions - possibly representing primary lysosomes. A phagocytosis assay with experimentally injected carbon particles revealed that both granular cell types displayed phagocytic properties, while the small lymphocyte-like cells did not show any evidence for this capacity (\cite{Moore1977}).

The morphological and ultrastructural findings of Moore and Lowe (1977) have since been confirmed by several investigators (\cite{Rasmussen1985, Renwartz1990, Pipe1990, Noel1994, Pipe1997, Wootton2003}). From their stand-alone electron microscopical examinations, Pipe and colleagues (1990) made a distinction between granular haemocytes with small (0.2-0.3 \micro m) and large (0.5-1.5 \micro m) granules. By relating the two ultrastructural phenotypes to their cytological staining properties, investigators later demonstrated that the two cell types corresponded to the basophilic and eosinophilic granular haemocytes of Moore and Lowe (\cite{Pipe1990, Noel1994}). Thus, if reduced to its static morphological criteria, Moore and Lowe's classification of \emph{M. edulis} haemocytes coincides with the original system of Cúenot (1891). This system generally recognized three types of haemocytes in bivalves: "(1) finely granular haemocytes, (2) coarsely granular haemocytes and (3) cells with very little cytoplasm surrounding the nucleus" (\cite{Cheng1984}). 

Leaning towards a phylum-wide two-categorical classification (hyalinocytes and granulocytes), Cheng (1981) argued that a distinction between the basophillic and eosinophilic granulocytes of \emph{M. edulis} was artificial, as he saw them as being immature and mature stages of the same cell type (granulocytes), respectively. From observations of what resembled (1) intermediate stages between the lymphocyte-like and larger basophilic cells and (2) a few smaller eosinophilic granulocytes (5-7 \micro m), Moore and Lowe (1977) had argued that the basophilic cells constituted a separate ontogenic lineage, with the larger phagocytic macrophages representing the final stage of maturation. This was further supported by observations of lymphocyte-like cells with mitotic figures, suggesting that it could be the stem cell of this lineage (\cite{Moore1977}). The notion that small lymphocyte-like basophils represented immature "prohemocyte" precursor cells was shared by Hine (1999), who argued that their basophilia indicated the presence of free ribosomes and immaturity, while their lack of cytoplasmic organelles precluded a secretory or phagocytic function.

Moore and Lowe's theory, as pointed out by Cheng (1984), was primarily formulated through interpretive evaluations of morphological findings, rather than being based on direct ontogenic evidence. The classification of bivalve haemocytes should ideally be constructed on the basis of their ontogeny. However, mapping of ontogenic lineages among bivalve haemocytes has been tempered by the lack of availible molecular databases, no one unifying model species, combined with uncertainty regarding the hematompoietic tissue(s) and processes of bivalves (\cite{Hine1999, Smith2016, Pila2016, delaBallina2022}). With no real ontogenic evidence to work with, a careful assessment of availible morphological data may represent a better alternative, relative to a classification based solely on biochemistry and function (\cite{Hine1999}). 

\subsection{Flow cytometric classification of \emph{M. edulis} haemocytes}
Almost two decades after flow cytometers became commercially availible in the 1970s (\cite{Shapiro2004}), the application of these instruments started to gain traction within the field of invertebrate immunopathology (\cite{Fisher1988}). Since the traditional characterization of bivalve haemocytes were largely based on morphological criteria such as size, granularity and staining affinities, the simultaneous measurement of forward scatter (\acrshort{fsc}, $\approx$ size) and side scatter (\acrshort{ssc}, internal complexity $\approx$ granularity) represented a far less subjective approach to their characterization (\cite{AshtonAlcox1998, Allam2002, Mateo2009}).

A detailed flow cytometric characterization of the haemocytes of \emph{M. edulis} was undertaken by Le Foll et al., (2010), who were able to distinguish three subpopulations according to cell size and \acrshort{ssc}. These comprised one population of small cells (7.14$\pm{0.05}$ \micro m) with low \acrshort{ssc}, one population of larger cells (9.97$\pm{0.17}$ \micro m) with intermediate \acrshort{ssc} and one population of large cells (10.08$\pm{0.24}$ \micro m) with high \acrshort{ssc}. By running haemocytes stained with eosin - which is fluorescent in the green/yellow spectrum (\cite{Elfer2016, Koegle2020}) - their results suggested that the latter subpopulation corresponded to eosinophilic granulocytes. However, these results were not visually verified by microscopy. The use of flow cytometers equipped with cell sorting capabilities simplifies the process of verifying any classification derived from flow cytometric measurements (\cite{Shapiro2004}). However, when extracting cells with known measured characteristics is not possible, the cells to be classified can be separated by other means prior to flow cytometric acquisition.

Since the three haemocyte cell types of \emph{M. edulis} differ with regard to the size and density of their granules, researchers managed to physically separate the eosinophilic granulocytes from the two basophilic cell types by isopycnic centrifugation (\cite{Friebel1995, Pipe1997}). Depending on the fixative used, the whole haemocyte population separated into three or four distinct cell-bands in the interfaces of the gradient layers. The two basophillic cell types could be isolated in high purity from the upper cell band (lowest density), the eosinophilic granulocytes from the lower, while the intermediate fractions often consisted of varying proportions of all three cell types. Accompanied by the rapid growth of flow cytometric applications in invertebrate immunology, the progress made by Friebel and Renwrantz (1995) and Pipe, Farley and Coles (1997) meant that light scatter profiles could be assigned to specific cell types individually.


\section{Cellular methods}
The cytological alterations accompanied by necrotic and programmed cell death (e.g., apoptosis) can be detected and quantified at high throughput with contemporary flow cytometers. While the two mechanisms share the same faith, the cytotoxic alterations that characterize them can be distinguished by several parameters - individually or by discriminating assays (\cite{Shapiro2003}). Since necrotic cell death is caused by damage to the cytoplasmic membrane, necrotic cells are primarily identified by loss of membrane integrity (\cite{Shapiro2003}). Conversely, apoptosis can have several proximate causes (\cite{Bedoui2020}), and the various structural and molecular alterations that proceed can be probed at different stages of the apoptotic process (\cite{Kari2022}). Allthough laser scanning cytometry is the preferred method for detecting apoptotic cells (\cite{Darzynkiewicz2001}), flow cytometric techniques have the advantage of being accessible, low-cost and highly versatile. In the present study, necrotic and apoptotic haemocytes were distinguished from viable cells by using the fluorescent dyes Calcein Acetoxymethyl ester (\acrshort{calceinam}) and TO-PRO$^{TM}$-3 Iodide (ToPro3) and the fluorogenic peptide; Apo-15.

\subsection{Calcein acetoxymethyl ester}
Calcein acetoxymethyl ester (\acrshort{calceinam}) is a cell-permeable esterase substrate commonly employed in viability assays and live cell imaging (\cite{Ramirez2010}). Similar to other electrically neutral fluorescein-derivatives, this small non-fluorescent molecule readily enters living cells through diffusion; where it is effectively trapped by intact cell membranes upon ester hydrolysis (\cite{Kaneshiro1993}). The retained poly-anionic calcein molecule is highly fluorescent in the green spectrum (\acrshort{exmax}/\acrshort{emmax}: 495/515 nm), and can be assayed by flow cytometry and epifluorescence microscopy (\cite{Wallach1959}, in: \cite{Chiu1977}). Due to its insensitivity to pH in the physiological range, high photo-stability and superior cellular retention (\cite{Chiu1977, Kaneshiro1993}); Calcein AM is a versatile probe with several advantages over other commonly employed esterase substrates (\cite{Ramirez2010}).

One critical drawback of employing calcein in standardized \emph{in vitro} viability assays resides in the fact that the acetoxymethyl tetraester form is a substrate of mammalian P-glycoprotein (P-gp) efflux pumps (\cite{Liminga1995}). Since different cell lines vary with regard to P-gp activity and hence the rate of calcein accumulation, results may not be directly comparable across different cell lines (\cite{Ramirez2010}). On the other hand, this exact property has led to the utilization of \acrshort{calceinam} as a fluorescent probe for assaying P-gp inhibitors (\cite{Di2016}). The inhibition of P-gp pumping activity results in an increased accumulation of hydrolyzed calcein within cells, which can then be detected and quantified through its fluorescence (\cite{Tiberghien1996, Köhler2003}).

Efflux pumps analogous to the mammalian P-gp and Multidrug Resistance-related protein (MRP) are also expressed in various bivalve tissues (\cite{Luckenbach2008, Luedeking2005}), including haemocytes in \emph{M. edulis} (\cite{Rioult2014}). By using P-gp- and MRP-specific inhibitors, Rioult et al., (2014) demonstrated that a MRP-type transporter were responsible for pumping Calcein AM out of haemocytes in \emph{M. edulis}, and that this activity was higher and more inducible in eosinophilic granulocytes compared to the other two cell types. While MRP-induction assayed through Calcein AM has been suggested as a potential biomarker for marine biomonitoring programs (\cite{Rioult2014, Minier1998}), the resulting differential accumulation of calcein might also represent a novel parameter for flow cytometric differential haemocyte counts in the common blue mussel.

\subsection{Apo-15/TO-PRO-3 Iodide apoptosis assay}
The Annexin-V/Propidium Iodide (PI) assay is considered a gold standard for monitoring cell death progression \emph{in vitro} (reviewed in \cite{VanEngeland1998}). This is a simple and feasible assay that is based on the differential (1) integrity of plasma membranes and (2) distribution of phosphatidylserine (PS) between bilipid layers of viable, necrotic and apoptotic cells (\cite{Jiang2016}). In homeostasis, PS is actively translocated to the inner leaflet of the plasma membrane by flippases (\cite{Connor1992}). As apoptotic cell death is induced, these flippases are cleaved by Caspases 3/7, and a PS-externalizing scramblase is irreversibly activated by the same mechanism (\cite{Verhoven1995, Suzuki2013}). The consequent exposure of PS by apoptotic cells serves a ligand for phagocytic macrophages (\cite{Fadok1992}) and persists throughout the whole apoptotic process (\cite{Martin1995}).

Annexin-V is a fluorogenic peptide that binds specifically to externalized PS in a Ca$^{2+}$-dependent manner (\cite{Andree1990, Martin1995}). As the externalization of PS precludes the gradual loss of membrane integrity in early apoptotic cells, they are not stained by membrane impermeable dsDNA-binding dyes such as PI. But since internal PS is accessible through the permeabilized membranes of necrotic and late apoptotic cells, they are recognized as Annexin-V$^{+}$/PI$^{+}$ cells. Viable cells are double negative, while early apoptotic cells are stained with Annexin-V externally (Annexin-V$^{+}$/PI$^{-}$). To facilitate for multifluorescence measurements, PI can be replaced by other membrane impermeable dsDNA-binding dyes with far-red emission (e.g., ToPro3 or \acrshort{7aad}). (\cite{Jiang2016})

Apoptosis probes that bind extracellular target molecules have advantages over \acrshort{tunel} assays and Caspase-binding probes, as they do not require cells to be permeabilized prior to staining (\cite{DelBino1999}). But, since Annexin-V requires extracellular Ca$^{2+}$ (>1 mM) for optimal PS binding (\cite{Andree1990}), this assay is not applicable with Ca$^{2+}$-free buffers or anticoagulant buffers containing divalent metal-chelators (e.g., EDTA). This property impairs the practical utility of Annexin-V in flow cytometric analyses of bivalve haemocytes, due to their rapid aggregation in the presence of free Ca$^{2+}$ (\cite{Torreilles1999}). On the bright side; a small fluorogenic peptide that binds PS in a Ca$^{2+}$-independent manner has recently been developed (Apo-15) (\cite{Barth2020}). This new fluorogenic probe holds great potential for discriminating between viable, necrotic and early apoptotic haemocytes \emph{in vitro}.

\section{Objective}
The overall aim of the present study was to develop a hybrid flow cytometric version of the Micronucleus cytome assay (\cite{Bolognesi2012}) with haemocytes from the common blue mussel (\emph{Mytilus edulis}). This approach would conserve the conventional scoring of genotoxic biomarkers (\acrshort{mni} and \acrshort{nbuds}) by light microscopy, but was aimed to streamline the scoring of cytotoxic alterations (apoptotic and necrotic haemocytes) and the differential haemocyte count with a novel flow cytometric methodology. \newline

\noindent The overall aim was pursued by working towards the following objectives:

\begin{enumerate}
    \item To develop a reliable sampling technique for the extraction of haemolymph from the posterior adductor muscle of \emph{M. edulis}, such that the frequency of unsuccessful extractions and contaminating particles can be minimized.
    \item To compare availible anticoagulant buffers/techniques with regard to their ability to inhibit haemocyte aggregation - without interfering with the targeted endpoints.
    \item To characterize the different haemocyte cell types of \emph{M. edulis} by conventional cytological criteria.
    \item To test wether side scatter and the differential ability to efflux \acrshort{calceinam} can be used to distinguish these cell types in suspensions of living haemocytes.
    \item To develop a flow cytometric differential haemocyte count that estimates the relative proportions of haemocyte cell types discernible in haemocytes preparations stained with Giemsa.
    \item To develop and validate a flow cytometric assay for scoring necrotic haemocytes by \acrshort{calceinam}/TO-PRO$^{TM}$-3 Iodide.
    \item To test the suitability of scoring necrotic and apoptotic hameocytes by Apo-15/TO-PRO$^{TM}$-3 Iodide under wash-free conditions (i.e., without centrifugation steps).
\end{enumerate}